\chapter{Introdução}
\label{ch:introducao}

Microsserviços são um padrão de arquitetura que permitem que serviços trabalhem juntos para um único serviço. Cada aplicativo usa sua linguagem e processo, permitindo que os microsserviços tenham várias vantagens, como flexibilidade, extensibilidade, alta disponibilidade, implantações e desenvolvimento independentes, isolamento de falhas, governança descentralizada e gerenciamento de dados descentralizado \cite{wang2020smart}. Esse estilo de arquitetura tem sido utilizado para aprimorar sistemas monolíticos, principalmente evitando que as informações sejam perdidas quando o aplicativo for reiniciado ou ocorrer falha em um dos serviços, portanto, eliminando a dependência entre um serviço e outro.

Devido à agilidade com que os sistemas são desenvolvidos, implantados e entregues hoje, a qualidade nesses processos tornou-se um requisito relevante. Para alcançar essa qualidade, outras tecnologias e mecanismos podem contribuir para sistemas mais robustos baseados em microsserviços, visando minimizar riscos e agregar maior segurança a soluções como blockchain \cite{xu2019blendmas}, \cite{dilshan2020mschain}.

A tecnologia blockchain tem se destacado em arquiteturas de software envolvendo microsserviços, pois combina outras tecnologias como criptografia, gerenciamento de dados, rede e
mecanismos de incentivo para fornecer, executar e registrar transações entre partes desconhecidas \cite{xu2019architecture}. Blockchain é um livro-razao distribuído estruturado em uma lista de blocos vinculados anexados ao seu predecessor por meio de hashes criptográficos. Isso fornece integridade e imutabilidade das informações, o que significa que qualquer modificação invalida o blockchain \cite{wust2018you}.

Por outro lado, as soluções que englobam ambas as tecnologias
também destacam que o uso de microsserviços pode resolver os gargalos atuais do blockchain, como escalabilidade, ordem de execução, replicação de dados de nó completo e contratos inteligentes de estilo imperativo \cite{bandara2021saas}. Microsserviços e blockchain compartilham conceitos essenciais como descentralização e autonomia \cite{cstefan2020blockchain}. Enquanto a arquitetura baseada em microsserviços é composta por serviços independentes e distribuídos, o blockchain permite que as informações sejam armazenadas de forma distribuída, trazendo resiliência, operações descentralizadas e multipartidárias \cite{de2020building}. Ambas as tecnologias apresentam novos desafios ao projetar e implementar sistemas de software \cite{cstefan2020blockchain}.

\section{Motivação}
\label{sec:motivacao}

Exemplo citação~\cite{lamport1986latex}.
%Reference without a non-breaking space 

Exemplo de outra citação \citeonline{quatroautores}, també utilizada.

\section{Objetivos}
\label{sec:objetivos}

\subsection{Objetivo Geral}
\label{subsec:objetivo-geral}

\subsubsection{Subsubsection}
\label{subsubsec:subsubsection}

\subsection{Objetivos Específicos}
\label{subsec:objetivos-especificos}
	\begin{alineas}
	    \item Objetivo específico 1;
		\item Objetivo específico 2;
		\item Objetivo específico 3.
	\end{alineas}

\section{Estrutura do Trabalho}
\label{sec:estrutura-do-trabalho}